\documentclass[12pt]{article} 
\usepackage[utf8]{inputenc} 
\usepackage[russian]{babel} 
\usepackage{amsmath} 
\usepackage{hyperref} % Включить ссылки в PDF 
\numberwithin{equation}{section} % Изменить нумерацию формул (4) -> (4.2) 
\begin{document} 
\tableofcontents % Вставить содержание
\newpage 
\subsubsection{Лемма Бёрнсайда} 
Пусть группа G действует на конечном множестве D. Элементы ~$d_1$ и ~$d_2$ из D назовём эквивалентным, если ~$d_1$ = g~$d_2$ для некоторого g из G. Нетрудно видеть, что множество D под действием группы G распадается на классы эквивалентности, состоящие из попарно эквивалентных элементов. Число классов эквивалентности можно найти при помощи следующей леммы Бёрнсайда. 

\begin{subequations} 

\textbf{Лемма 4.2} 
\textit{Пусть группа G действует на конечном множестве D. Тогда для N - числа классов эквивалентности, порождаемых на множестве D действием группы G, справедливо равенство} 
$$N = \frac{1}{|G|} \sum \limits_{g\ni G} \psi(g),$$ 
\textit{где $\psi (g)$ - число элементов d множества D таких, что $gd = d$.} 
\label{eq.simple} 
\end{subequations} 

{\scshape Доказательство.} 
\textup{Введём функцию $\psi (d, g)$ так, что} 
\[ 
\psi(d, g) = 
\begin{cases} 
1, & \text{если $gd = d$;}\\ 
0, & \text{если $gd \neq d$.} 
\end{cases} 
\] 

\textit{Тогда учитывая предыдущую лемму,} 
{$ \sum \limits_{g \ni G} \psi(g) = \sum \limits_{g\ni G} \sum \limits_{d\ni D} \psi (d, g) = \sum \limits_{~O_{r_{i}}} \sum \limits_{d \ni ~O_{r_{i}}} \sum \limits_{g \ni G} \psi (d, g) $} 

{\rm Разделив левую и правую части получившегося равенства на |G|, получаем требуемую формулу для N. Лемма доказана.} 

\end{document}